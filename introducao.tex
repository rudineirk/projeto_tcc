\chapter{INTRODUÇÃO}
\label{chp:intro}

Desde que foi criado, o computador é utilizado para auxiliar em atividades humanas. Hoje, ele está presente e se faz necessário em qualquer área de aplicação: seja em bancos para efetuar cálculos matemáticos, em hospitais, exibindo dados para monitoramento do estado clínico dos pacientes, ou até mesmo no trânsito, onde por meio de visão computacional é possível identificar o número das placas dos veículos que trafegam em alta velocidade.

Uma das áreas de aplicações que os computadores vêm sendo largamente utilizados atualmente, é na visão computacional, mais especificamente na área de reconhecimento de faces. A identificação de faces humanas utilizando-se de visão computacional, é suportada por ferramentas que facilitam no desenvolvimento de várias aplicações. Estas ferramentas são de grande importância, no entanto, exigem dos usuários grande esforço para utilizá-las. Mesmo sendo disponibilizado um conjunto de bibliotecas com funções pré-definidas, ainda assim é preciso ter conhecimento prévio a fim de implementá-las em algum sistema computacional.

Hoje já existem técnicas de engenharia de software que permitem a criação de unidades independentes de sistemas, que proveem um conjunto de interfaces mais amigáveis onde há a necessidade de saber como interagir com elas. Estas unidades independentes são conhecidas como componentes. 

Tendo em vista a grande gama de aplicações das ferramentas de visão computacional, e o seu crescente uso para os mais diversos fins, como poderia se desenvolver um componente capaz de abstrair essas funcionalidades e disponibilizá-las de forma mais fácil de serem utilizadas?

Acredita-se que, a utilização de visão computacional em conjunto com técnicas para reconhecimento de padrões e aprendizagem de máquina, possibilitem a criação de um componente de software capaz de abstrair estas características e disponibilizá-las por meio de interfaces para a identificação de pessoas.

\cite{LiJain2011} afirmam que, basta olhar ao redor, e se comprovará que sistemas de visão computacional estão sendo implementados em todo lugar: estão em carros para ajudar a estacionar em locais estreitos, nos laptops como software de reconhecimento facial para segurança adicional, no Facebook e Google+ utilizam de visão computacional para identificar as pessoas em álbuns de fotos.

Demonstrada a grande variedade de aplicações para visão computacional, tem-se a preocupação com uma parcela de pessoas, principalmente na área de tecnologia, que não têm conhecimentos necessários para poder utilizar desta tecnologia para a automatização de processos. Desta forma, tem-se como objetivo geral desenvolver um componente de software que possibilite a utilização de visão computacional por aplicativos de terceiros. Considera-se terceiros, as pessoas em que este componente é de interesse, seja para uso comercial ou não.

Para que se possa alcançar o objetivo geral do projeto, será preciso atender os seguintes objetivos específicos: levantar requisitos para elaboração do componente de software; elaborar modelos baseando-se em engenharia de software, que descreverão as funcionalidades do componente; desenvolver o componente de software; treinar o componente para reconhecimento de faces; validar as funcionalidades do componente por meio de um estudo de caso; apresentar os resultados obtidos.

A metodologia a ser empregada no presente projeto, com o intuito de verificar a sua viabilidade, é do tipo exploratória. Desta forma, para a aquisição de novos conhecimentos, serão considerados várias fontes para coleta de dados, como: livros, artigos, trabalhos acadêmicos, web sites, etc.

Este trabalho está dividido em 3 capítulos, no capítulo 1 são abordados conceitos de engenharia de software, dando ênfase à importância da reusabilidade ao se criar um componente de software, bem como os modelos para a definição de componentes. No capítulo 2, são apresentados conceitos relacionados a inteligência artificial, aprendizagem de máquina e algumas técnicas acerca destes temas. No capítulo 3, são explanados os processos que envolvem a detecção e reconhecimento de faces, dando maior importância aos Haar cascades e Eigenfaces que serão aplicados neste trabalho.
