\chapter{INTRODUÇÃO}
\label{chp:intro}

Os administradores de sistema utilizam como ferramenta de trabalho interfaces
de administração de servidores. Existem empresas que fornecem estas interfaces
de administração como produtos, os quais têm como principal foco expor as
funcionalidades dos serviços do servidor, de forma que facilite a tarefa de
administração. Essas ferramentas são essenciais para o bom desempenho do
trabalho desses profissionais. Caso aconteça uma falha na ferramenta, isto
pode ocasionar em erros no serviço administrado, que podem gerar retrabalho
e custos diversos.

Esse trabalho se baseia no desenvolvimento da segunda iteração de uma
interface de administração de servidores. O projeto desta segunda versão
foi iniciado sem se basear no código existente da primeira versão, por esta
apresentar alguns problemas críticos estruturais. Isto se deve ao fato da
empresa responsável pelo produto ter iniciado como uma terceirizada de
administração de serviços, mas em certo ponto moveu seu foco para o
desenvolvimento de um produto próprio. Esse produto iniciou com base em
\emph{scripts} desenvolvidos antes da empresa ter funcionários dedicados a
desenvolvimento, sem os cuidados devidos de estruturação de código e escolhas
de tecnologias, sendo estas as principais causa dos problemas estruturais
existentes no produto. Esse trabalho procura responder a seguinte pergunta:
É possível criar uma interface de administração de servidores utilizando
práticas modernas de desenvolvimento de software?

Segundo \citeonline{Hochstadt2006}, a reescrita de um sistema que teve seu
início baseado em \emph{scripts} nem sempre resolve problemas existentes,
além de poder criar problemas que não existiam antes. A forma correta de
aproximar um problema de manutenção de um sistema legado é a reanálise das
regras de negócio associadas ao mesmo, para então ser tomada a decisão de
correção do sistema antigo ou reescrita deste. Para este projeto, a análise
foi realizada previamente e foi decidido pela reescrita do sistema, mas esta
análise não faz parte do escopo deste trabalho.

O desenvolvimento de uma interface de administração de servidores Linux
envolve diversas tecnologias e uma gama de serviços que devem ser
administrados, isto gera uma base de código relativamente grande, o que
dificulta o isolamento dos domínios de aplicação na arquitetura
do sistema. Segundo \citeonline{Newman2015}, isto ocorre em todas as aplicações
que utilizam uma arquitetura monolítica, em escalas variáveis, mas sempre
geram problemas como mistura de código de domínios diferentes, dificultando
a correção de \emph{bugs} e implementação de novas funcionalidades.

Segundo \citeonline{Fowler2016}, arquiteturas de microsserviços reduzem
alguns dos problemas provenientes do aumento da complexidade de uma aplicação
por meio da divisão da aplicação em pequenos serviços com uma camada leve
de comunicação entre si. O isolamento de domínios da aplicação diminui a
inércia do fluxo de entrega de software funcional, uma vez que podem ser
feitas entregas pequenas e rápidas, diminuindo assim o \emph{Time to Market}.

O principal objetivo deste projeto é desenvolver uma interface de
administração de servidores modular, utilizando técnicas de desenvolvimento
de software modernas. Para tal, são realizadas as seguintes etapas que compõe
a execução do mesmo:

\begin{alineas}
  \item analisar arquiteturas de software e técnicas de estruturação \\
    de microsserviços;
  \item escolher tecnologias que serão utilizadas;
  \item desenhar a arquitetura inicial da interface de administração de \\
    servidores modular;
  \item desenvolver o protótipo da aplicação da arquitetura;
  \item realizar um estudo de caso da aplicação da arquitetura escolhida.
\end{alineas}

Para a execução deste trabalho são aplicadas duas metodologias, a primeira
é a pesquisa bibliográfica, onde é realizado a consulta de referencial
teórico e estudo de casos de sucesso no uso de arquiteturas de microsserviços.
A segunda parte é um estudo de caso, onde as informações coletadas
na primeira etapa são utilizadas para elaboração da arquitetura da interface
de administração de servidores modular e para desenvolvimento de um protótipo.

Esse trabalho está dividido em cinco capítulos, no primeiro capítulo é
apresentado as motivações e os objetivos deste trabalho. No capítulo
\ref{chp:tecnologias} são apresentados os principais conceitos e tecnologias
envolvendo microsserviços, abordando um pouco sobre arquiteturas
e práticas que originaram o conceito de microsserviços. O capítulo
\ref{chp:arquitetura} se trata do desenho da arquitetura do sistema. No
capítulo \ref{chp:resultados} é apresentando os resultados obtidos na execução
do projeto e dificuldades encontradas. Por último a conclusão do projeto,
onde é apresentado uma revisão geral do trabalho e dos resultados encontrados.
