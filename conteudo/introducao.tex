\chapter{INTRODUÇÃO}
\label{chp:intro}

Para administradores de sistema, as ferramentas as quais eles utilizam são
críticas para a execução do seu trabalho, onde uma falha pode parar a execução
de algum serviço crítico da empresa na qual ele trabalha, gerando custos para
a mesma. Para administração destes serviços, existem empresas que fornecem
produtos que provem interfaces de administração dos serviços, os quais tem
como principal foco expor as funcionalidades dos serviços de forma que
facilite a tarefa de administração.

Este trabalho se baseia no desenvolvimento da segunda iteração de uma
interface de administração de servidores. O projeto desta segunda versão
foi iniciado sem se basear no código existente da primeira versão, por esta
apresentar alguns problemas críticos estruturais. Isto se deve ao fato da
empresa responsável pelo produto ter iniciado como uma tercerizada de
administração de serviços, mas em certo ponto moveu seu foco para o
desenvolvimento de um produto próprio. Este produto, o objetivo deste
trabalho, iniciou-se com base em \emph{scripts} desenvolvidos antes da empresa
ter funcionários dedicados a desenvolvimento, sem os cuidados devidos de
estruturação de código e escolhas de tecnologias, sendo estas as principais
causa dos problemas estruturais existentes no produto.

O desenvolvimento de uma interface de administração de servidores Linux
envolve diversas tecnologias e uma gama de serviços que devem ser
administrados, isto gera uma base de código relativamente grande, o que
dificulta o isolamento dos domínios de aplicação na arquitetura
do sistema. Segundo \citeonline{Newman2015}, isto ocorre em todas as aplicações
que utilizam uma arquitetura monolítica, em escalas variáveis, mas sempre
geram problemas como mistura de código de domínios diferentes, dificultando
a correção de bugs e implementação de novas funcionalidades. Este trabalho
procura responder a seguinte pergunta: É possível criar uma interface de
administração de servidores utilizando práticas modernas de desenvolvimento
de software?

Segundo \citeonline{Fowler2016}, arquiteturas de microserviços reduzem
alguns dos problemas provenientes do aumento da complexidade de uma aplicação
por meio da divisão da aplicação em pequenos serviços com uma camada leve
de comunicação entre si.

Ao contrário de aplicações tradicionais utilizando arquitetura
\ac{SOA}, que expõe interfaces \ac{RPC} detalhadas que simulam o funcionamento
de uma arquitetura monolítica, segundo \citeonline{Erl2008}, são expostas
\ac{API} no estilo \ac{REST} que facilitam o uso por outras aplicações e
por terceiros.

O isolamento de domínios da aplicação diminui a inércia do fluxo de entrega
de software funcional, uma vez que podem ser feitas entregas pequenas e
rápidas, diminuindo assim o \emph{Time to Market}.

O principal objetivo deste projeto é o desenvolvimento de uma interface
de administração de servidores modular, utilizando técnicas de desenvolvimento
de software modernas. Para tal, são realizadas as seguintes etapas que compõe
a execução do mesmo:

\begin{alineas}
  \item analisar arquiteturas de software e técnicas de estruturação \\
    de microserviços;
  \item escolher tecnologias que serão utilizadas;
  \item desenhar a arquitetura inicial da interface de administração de \\
    servidores modular;
  \item desenvolver o protótipo da aplicação da arquitetura;
  \item realizar um estudo de caso da aplicação da arquitetura escolhida.
\end{alineas}

Para a execução deste trabalho são aplicadas duas metodologias, a primeira
é a pesquisa bibliográfica, onde é realizado a consulta de referencial
teórico e estudo de casos de sucesso no uso de arquiteturas de microserviços.

A segunda parte é um estudo de caso, onde as informações coletadas
na primeira etapa são utilizadas para elaboração da arquitetura da interface
de administração de servidores modular e para desenvolvimento de um protótipo.

Este trabalho está dividido em cinco capítulos, no primeiro capítulo é
apresentado as motivações e os objetivos deste trabalho. No capítulo
\ref{chp:tecnologias} são apresentados os principais conceitos e tecnologias
envolvendo microserviços, abordando um pouco sobre arquiteturas
e práticas que originaram o conceito de microserviços. O capítulo
\ref{chp:arquitetura} se trata do desenho da arquitetura do sistema. No
capítulo \ref{chp:resultados} é apresentando os resultados obtidos na execução
do projeto e dificuldades encontradas. Por último a conclusão do projeto,
onde é apresentado uma revisão geral do trabalho e dos resultados encontrados.
