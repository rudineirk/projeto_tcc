\chapter{MICROSERVIÇOS}

Microserviços são um formato de desenvolvimento de sistemas que trata o
problema de isolamento dos domínios da aplicação por meio da distribuição
da lógica entre pequenos processos isolados que trabalham em conjunto para a
entrega da funcionalidade final esperada.

Segundo \citeonline{Newman2015}, esta nova forma de trabalho otimiza o processo
de desenvolvimento, uma vez que os microserviços normalmente tem sua base de
código separada dos outros serviços, permitindo que o desenvolvedor realize
alterações sem o risco de afetar outras áreas da aplicação.

\section{ARQUITETURAS}

Historicamente houveram diversas tentativas de criar arquiteturas de softwares
resilentes e manteníveis, muitas as quais sofreram com a ação de novas de
tecnologias disponíveis, como acesso a virtualização instantânea de baixo custo,
e algumas mudanças de paradigmas de estruturação dos projetos (maior foco na
escalabilidade do que na otimização de baixo nível).

\subsection{Monolítica}

O modelo clássico de desenvolvimento de qualquer projeto é composto por um
único processo responsável pela entrega de todas as funcionalidades de um
produto. Empresas como Google, Amazon e Netflix utilizaram por muito tempo
esta arquitetura, principalmente pelo baixo custo de infraestrutura inicial,
tendo em vista que o custo de manutenção envolve somente a manutenção deste
único binário e de seus dados.

Segundo \citeonline{Erl2008}, este modelo é válido e recomendado para novos
projetos, considerando que nem sempre as linhas de separação entre as regras
de negócio estão bem definidas. Este processo ocorre naturalmente durante o
ciclo de vida de um projeto, e muitas vezes a separação a nível de código já
é suficiente a nível de isolamento de domínios.

Os problemas deste formato iniciam conforme a base de código cresce e a demanda
das funcionalidades aumentam. Uma grande base de código que deve ser empacotada
e levada a produção de uma única vez cria uma grande inércia no desenvolvimento
de novas funcionalidades, pois alterações em uma funcionalidade ter sua
liberação atrasada por causa de alterações em outras partes do código não
relacionadas, mas que fazem parte do mesmo repositório de código.

Esta arquitetura atende muito bem alguns tipos de produtos, mas aplicações
expostas para internet que podem ter milhares de acessos simultâneos expõe o
principal ponto fraco deste formato: escalabilidade. Este problema ocorre
quando uma funcionalidade do monolíto tem um aumento de demanda maior que o
limite físico de uma única máquina, sendo necessário alocar outra instância
da aplicação, mas para isto é necessário alocar recursos para todas as
funcionalidades, uma vez que o software é composto de um único binário. Estes
recursos alocados desnecessariamente aumentam muito o custo de operação e
manutenção do sistema, o que reflete diretamente no negócio que depende deste.

\subsection{SOA}

Para criar o isolamento das aplicações necessário para atender as novas
necessidades do mercado, foram criados um conjunto de conceitos que englobam o
que é chamado de \ac{SOA}. Estes conceitos definem algumas regras e diretrizes
para isolamento de domínios de aplicações e comunicação entre as diversas partes
do sistema.

Segundo \citeonline{Erl2008}, este formato de arquitetura segue alguns
princípios que determinam a classificação de um sistema como \ac{SOA},
os quais são:

\begin{enumerate}
  \item Contratos de comunicação padronizados
  \item Baixo nível de acoplamento
  \item Abstração da regra de negócio
  \item Reusabilidade de um serviço
  \item Autonomia de um serviço
  \item Inexistência de manutenção de estados
  \item Descoberta de serviços
  \item Componibilidade de serviços
\end{enumerate}

Os contratos de comunicação padronizados se referem ao estabelecimento de
formatos de dados e métodos de interação entre os serviços. Um dos padrỗes
que tiveram um uso elevado é o \ac{SOAP}, que define um formato padronizado
de comunicação de aplicações utilizando dados no formato \ac{XML}. Este padrão
era flexível, e permitia a criação de implementações mais específicas, como o
\ac{WSDL}, que implementa um protocolo \ac{RPC} utilizando este padrão como
meio.

O baixo nível de acoplamento engloba o isolamento de domínios de aplicação e as
interdependências entre serviços, os quais não devem depender da forma como são
implementados outros serviços. Isto conflita em alguns pontos com contratos de
comunicação utilizando \ac{RPC}, pois o mesmo simula o funcionamento de um
sistema monolítico.

Um serviço não deve expor contratos de comunicação que dependam da forma como
ele são implementados, esta é a definição básica de abstração da regra de
negócio, na qual um serviço deve expor funcionalidades e não a sua
implementação.

A reusabilidade de um serviço deve-se ao fato que existe poucas vantagens em
isolar uma regra de negócio utilizada exclusivamente por um único serviço. Uma
regra de negócio deve se tornar um serviço isolado somente se ela for utilizada
por mais sistemas, caso contrário este serviço estaria somente adicionando
complexidade ao sistema.

Falhas em partes da aplicação são um risco que não pode ser mitigado de forma
simples, pois diversos fatores externos a aplicação podem comprometer o bom
funcionamento de um dos serviços, mas isto não deve comprometer o
funcionamento de outras partes do sistema. No caso de sistemas monolíticos
uma falha em uma parte da aplicação poderia matar o processo e indisponibilizar
a aplicação como um todo, no caso de sistemas implementados com \ac{SOA}, os
serviços devem continuar a disponibilizar, na medida do possível, suas
funcionalidades mesmo que um serviço o qual este depende apresente algum tipo
de problema.

Manter estado de um cliente de um serviço consome recursos e tempo de
computação que podem ser evitados caso não seja mantido estados. Para tal,
cada requisição na comunicação deve carregar todas as informações necessárias
para realizaçao da operação. Este conceito de comunicação é aplicado em diversos
protocolos de comunicação assincronos, como por exemplo o \ac{HTTP}, os quais
por limitações físicas não podem manter estados a fim aumentar sua resilência
em casos de falhas de comunicação.

Um sistema composto por diversos serviços deve conseguir descobrir onde os
serviços estão rodando e como acessa-los. Isto pode ser feito por meio de um
registro de serviços, que armazena uma lista de serviços atualmente em execução.
Esta funcionalidade também é utilizada para alta disponibilidade e distribuição
de carga, uma vez que este registro pode apontar para mais de uma instância do
mesmo serviço. Um exemplo prático das vantagens deste princípio é uma manobra
de manutenção em um servidor, onde pode ser removido o apontamento de uma
instância do registro de serviços sem que haja paradas no sistema.

Por fim o conceito que envolve uma alteração grande de paradigmas é a
componibilidade de serviços, onde sistemas são construidos utilizando serviços
como blocos de montagem, nos quais as camadas de serviço podem depender de
outras camadas de serviços. Isto permite o reuso de lógica de aplicações
completamente diferentes para compor uma nova aplicação, o que é possível
somente se outros princípios como o baixo acoplamento e isolamento da regra de
negócio forem aplicados corretamente. Uma grande vantagem da aplicação deste
conceito é que diferentes aplicações podem compartilhar de uma mesma
infraestrutura de serviços.

\subsection{Microserviços}

\section{REST}

\section{MESSAGE QUEUES}

\subsection{Brokered}

\subsection{Brokerless}

\section{BALANCEAMENTO DE CARGA}

\section{DESCOBERTA DE SERVIÇOS}

\section{CONTAINERS LINUX}

\section{ORQUESTRAÇÃO DE MICROSERVIÇOS}

\section{BANCO DE DADOS}

\subsection{NoSQL}
