\chapter*{RESUMO}

\noindent Arquiteturas de \emph{software} são um campo complexo de ser estudado,
pois envolvem não somente aspectos tecnológicos, mas também problemas
humanos, como conhecimento de conceitos e concepções utilizadas para
aplicação e manutenção de um sistema desenvolvido na arquitetura em questão.
Microsserviços são um exemplo de uma arquitetura que tenta resolver alguns
problemas recorrentes em outras arquiteturas relacionados a carga de
conhecimento sobre um certo sistema requerida para alterar ou adicionar uma
nova funcionalidade ao mesmo. Para tal, é realizado o isolamento de domínios da
aplicação em serviços, sendo cada serviço é responsável somente por sua própria
regra de negócio. Como estas regras normalmente não trabalham totalmente
isoladas umas das outras, é implementado alguns formatos de comunicação entre
estes serviços, a fim de prover a funcionalidade desejada pelo usuário final.
Este trabalho procura estudar e aplicar esta arquitetura para o
desenvolvimento de uma interface de administração de servidores. Durante o
desenvolvimento do trabalho, são listadas algumas alternativas de
tecnologias e componentes estudados para implementação do sistema,
identificando quais não atenderam os requisitos e quais foram aplicadas
durante a implementação da arquitetura. Em seguida é apresentado a
arquitetura final, demonstrando os componentes e a interação entre estes
para expor as funcionalidades do sistema. Por fim, são apresentados os
resultados do projeto, onde é descrito as facilidades de gerenciamento,
manutenção do código, escalabilidade e tolerância a falhas que foram
implementadas, além de melhorias futuras do sistema.

\newline\newline\noindent \textbf{Palavras-chave:} \ptBRKeyword
