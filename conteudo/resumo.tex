\chapter*{RESUMO}

\noindent Arquiteturas de \emph{software} é um campo complexo de ser estudado,
pois envolvem não somente aspectos tecnológicos, mas também problemas
humanos, como conhecimento de conceitos e concepções utilizadas para
aplicação e manutenção de um sistema desenvolvido na arquitetura em questão.
Microsserviços são um exemplo de uma arquitetura que tenta resolver alguns
problemas recorrentes em outras arquiteturas relacionados a carga de
conhecimento sobre um certo sistema requerida para alterar ou adicionar uma
nova funcionalidade ao mesmo. Este trabalho procura aplicar esta
arquitetura para o desenvolvimento de uma interface de administração de
servidores. Durante o desenvolvimento do trabalho, são listadas algumas
alternativas de tecnologias e componentes estudados para implementação
do sistema, identificando quais não atenderam os requisitos e quais
foram aplicadas durante a implementação da arquitetura. Em seguida é
apresentado a arquitetura final, demonstrando os componentes e a
interação entre estes para expor as funcionalidades do sistema. Por
fim, são apresentados os resultados do projeto, onde são descritas as
facilidades de gerenciamento, manutenção do código, escalabilidade e
tolerância a falhas que foram implementadas, além de melhorias futuras
do sistema.

\newline\newline\noindent \textbf{Palavras-chave:} \ptBRKeyword
