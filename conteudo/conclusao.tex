\chapter{CONCLUSÃO}
\label{chp:conclusao}

Este trabalho teve como objetivo a criação de uma interface de administração
de servidores baseada em arquitetura de microsserviços. A necessidade do
mesmo surgiu com base em um planejamento feito para a nova iteração de um
produto de administração de servidores, que sofria com problemas de
dificuldade de manutenção e adição de novas funcionalidades. O projeto
procurou implementar esta nova versão com uma base sólida de tecnologias,
utilizando práticas modernas de desenvolvimento de \emph{software}, tais como
a arquitetura de microsserviços.

O desenvolvimento de um projeto utilizando a arquitetura de microsserviços
pode se provar como um grande desafio, pois envolve uma grande quantidade
de tecnologias e conceitos. Conceitos de sistemas distribuídos são aplicados
de forma concreta nesta arquitetura, por se tratar da distribuição das regras
de negócio em aplicações dedicadas que se comunicam entre si, para prover
a funcionalidade final.

Na aplicação da arquitetura, foram utilizados ferramentas disponíveis e
provadas, como Docker e Swarm para gerenciamento e distribuição dos serviços,
o que se provou na prática como uma excelente escolha. Isto se deve a
estrutura de comunicação, que utiliza RabbitMQ, que prove uma arquitetura
resiliente e escalável, sem aumentar significativamente o consumo de recursos
da máquina de execução do sistema.

As dificuldades encontradas foram principalmente relacionadas ao grande leque
de tecnologias, formas de comunicação entre serviços e estruturação dos
componentes. A comunicação em especial se provou como o ponto mais complexo
da arquitetura de microsserviços, por se tratar de uma mudança de paradigma
em relação a estrutura comum de estruturação de aplicações, que é síncrona.
Para comunicação entre microsserviços, é necessário a utilização de protocolos
assíncronos, como o \ac{AMQP}, para tirar proveito das melhores vantagens
da arquitetura.

O desenvolvimento de novos serviços, sem o auxílio de bibliotecas de apoio,
para abstrair estes paradigmas assíncronos, pode ser desafiador. Para isto,
a estrutura interna dos microsserviços foi estruturada para isolar as regras
de negócio da comunicação com outros componentes, como a interação
com o \ac{MQ} e com o banco de dados. Esta arquitetura interna dos serviços
se provou funcional, mas houve problemas relacionados a testes unitários
e o bloqueamento de recursos, que ocorreram devido ao acoplamento e a
comunicação síncrona entre os componentes internos dos serviços.

Para a continuidade deste trabalho, está planejado a alteração da arquitetura
interna dos serviços, passando a utilizar atores, que são componentes que
se comunicam de forma assíncrona. A utilização deste novo formato deve
eliminar problemas de bloqueamento da instância do serviço durante uma
requisição e reduzir o consumo de recursos do sistema como um todo. Do
ponto de vista de produto, para a disponibilização do objeto deste trabalho
para o mercado, ainda é necessário o desenvolvimento dos serviços que
proveram as funcionalidades esperadas, utilizando a arquitetura desenvolvida
para este trabalho, além de uma interface gráfica para o usuário final.
