\chapter{CONCLUSÃO} \label{sec:capitulos}

A gestão por competências dentro de uma organização permite a visualização das reais necessidades das capacidades técnicas e comportamentais para que a empresa invista de maneira correta seus recursos. Por outro lado, apenas a aplicação de um modelo de competências não seja o suficiente para que funcionários adquiram o máximo de sua eficiência.

Para uma completa solução em gestão de funcionários a aplicação da gestão do conhecimento junto a gestão de competências se faz necessário para que haja o compartilhamento das habilidades já existentes e também ao aprimoramento das habilidades ainda necessárias para que o funcionário atinja um nível satisfatório de desempenho em suas atividades.

Este artigo apresentou uma proposta de modelo por competências para auxiliar gestores e profissionais de Recursos humanos na obtenção e monitoramento de funcionários em uma empresa metal mecânica. Desta forma foram descritos os conceitos necessários para entendimento dos assuntos, Gestão do conhecimento e Gestão por competências, e também a apresentação de uma ideia de modelo por competências.

No sentido de que o setor de manufatura na maioria das empresas possui profissionais preparados ou que por intermédio de incentivos da empresa poderiam buscar o conhecimento necessário para estarem qualificados. Uma ferramenta computacional dotada de um mapeamento consistente com as reais necessidades da empresa torna-se um potente recurso motivador para os empregados e estratégico para empresa que busca a competitividade no mercado.

Com a correta estratégia colocada em pratica e com a sua execução para obtenção de um mapeamento de competências, pode-se pensar futuramente no desenvolvimento de um modelo de ferramenta computacional preparada para gerir as competências técnicas e comportamentais de funcionários alocados no ambiente industrial de uma empresa.
