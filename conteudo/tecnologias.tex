\chapter{TECNOLOGIAS}
\label{chp:tecnologias}

Para o desenvolvimento deste projeto, são necessárias diversas tecnologias e
conceitos, os quais envolvem várias áreas da computação. Neste capítulo são
abordados os principais assuntos relacionados ao foco principal deste documento.

\section{ARQUITETURAS}

Historicamente houveram diversas tentativas de criar arquiteturas de softwares
resilentes e manteníveis, muitas as quais sofreram com a ação de novas de
tecnologias disponíveis, como acesso a virtualização instantânea de baixo custo,
e algumas mudanças de paradigmas de estruturação dos projetos (maior foco na
escalabilidade do que na otimização de baixo nível).

\subsection{Monolítica}

O modelo clássico de desenvolvimento de qualquer projeto é composto por um
único processo responsável pela entrega de todas as funcionalidades de um
produto. Empresas como Google, Amazon e Netflix utilizaram por muito tempo
esta arquitetura, principalmente pelo baixo custo de infraestrutura inicial,
tendo em vista que o custo de manutenção envolve somente a manutenção deste
único binário e de seus dados.

Segundo \citeonline{Erl2008}, este modelo é válido e recomendado para novos
projetos, considerando que nem sempre as linhas de separação entre as regras
de negócio estão bem definidas. Este processo ocorre naturalmente durante o
ciclo de vida de um projeto, e muitas vezes a separação a nível de código já
é suficiente a nível de isolamento de domínios.

Os problemas deste formato iniciam conforme a base de código cresce e a demanda
das funcionalidades aumentam. Uma grande base de código que deve ser empacotada
e levada a produção de uma única vez cria uma grande inércia no desenvolvimento
de novas funcionalidades, pois alterações em uma funcionalidade ter sua
liberação atrasada por causa de alterações em outras partes do código não
relacionadas, mas que fazem parte do mesmo repositório de código.

Esta arquitetura atende muito bem alguns tipos de produtos, mas aplicações
expostas para internet que podem ter milhares de acessos simultâneos expõe o
principal ponto fraco deste formato: escalabilidade. Este problema ocorre
quando uma funcionalidade do monolíto tem um aumento de demanda maior que o
limite físico de uma única máquina, sendo necessário alocar outra instância
da aplicação, mas para isto é necessário alocar recursos para todas as
funcionalidades, uma vez que o software é composto de um único binário. Estes
recursos alocados desnecessariamente aumentam muito o custo de operação e
manutenção do sistema, o que reflete diretamente no negócio que depende deste.

\subsection{SOA}

Para criar o isolamento das aplicações necessário para atender as novas
necessidades do mercado, foram criados um conjunto de conceitos que englobam o
que é chamado de \ac{SOA}. Estes conceitos definem algumas regras e diretrizes
para isolamento de domínios de aplicações e comunicação entre as diversas partes
do sistema.

Segundo \citeonline{Erl2008}, este formato de arquitetura segue alguns
princípios que determinam a classificação de um sistema como \ac{SOA},
os quais são:

\begin{enumerate}
  \item Contratos de comunicação padronizados
  \item Baixo nível de acoplamento
  \item Abstração da regra de negócio
  \item Reusabilidade de um serviço
  \item Autonomia de um serviço
  \item Inexistência de manutenção de estados
  \item Descoberta de serviços
  \item Componibilidade de serviços
\end{enumerate}

Os contratos de comunicação padronizados se referem ao estabelecimento de
formatos de dados e métodos de interação entre os serviços. Um dos padrỗes
que tiveram um uso elevado é o \ac{SOAP}, que define um formato padronizado
de comunicação de aplicações utilizando dados no formato \ac{XML}. Este padrão
era flexível, e permitia a criação de implementações mais específicas, como o
\ac{WSDL}, que implementa um protocolo \ac{RPC} utilizando este padrão como
meio.

O baixo nível de acoplamento engloba o isolamento de domínios de aplicação e as
interdependências entre serviços, os quais não devem depender da forma como são
implementados outros serviços. Isto conflita em alguns pontos com contratos de
comunicação utilizando \ac{RPC}, pois o mesmo simula o funcionamento de um
sistema monolítico.

Um serviço não deve expor contratos de comunicação que dependam da forma como
ele são implementados, esta é a definição básica de abstração da regra de
negócio, na qual um serviço deve expor funcionalidades e não a sua
implementação.

A reusabilidade de um serviço deve-se ao fato que existe poucas vantagens em
isolar uma regra de negócio utilizada exclusivamente por um único serviço. Uma
regra de negócio deve se tornar um serviço isolado somente se ela for utilizada
por mais sistemas, caso contrário este serviço estaria somente adicionando
complexidade ao sistema.

Falhas em partes da aplicação são um risco que não pode ser mitigado de forma
simples, pois diversos fatores externos a aplicação podem comprometer o bom
funcionamento de um dos serviços, mas isto não deve comprometer o
funcionamento de outras partes do sistema. No caso de sistemas monolíticos
uma falha em uma parte da aplicação poderia matar o processo e indisponibilizar
a aplicação como um todo, no caso de sistemas implementados com \ac{SOA}, os
serviços devem continuar a disponibilizar, na medida do possível, suas
funcionalidades mesmo que um serviço o qual este depende apresente algum tipo
de problema.

Manter estado de um cliente de um serviço consome recursos e tempo de
computação que podem ser evitados caso não seja mantido estados. Para tal,
cada requisição na comunicação deve carregar todas as informações necessárias
para realizaçao da operação. Este conceito de comunicação é aplicado em diversos
protocolos de comunicação assincronos, como por exemplo o \ac{HTTP}, os quais
por limitações físicas não podem manter estados a fim aumentar sua resilência
em casos de falhas de comunicação.

Um sistema composto por diversos serviços deve conseguir descobrir onde os
serviços estão rodando e como acessa-los. Isto pode ser feito por meio de um
registro de serviços, que armazena uma lista de serviços atualmente em execução.
Esta funcionalidade também é utilizada para alta disponibilidade e distribuição
de carga, uma vez que este registro pode apontar para mais de uma instância do
mesmo serviço. Um exemplo prático das vantagens deste princípio é uma manobra
de manutenção em um servidor, onde pode ser removido o apontamento de uma
instância do registro de serviços sem que haja paradas no sistema.

Por fim o conceito que envolve uma alteração grande de paradigmas é a
componibilidade de serviços, onde sistemas são construidos utilizando serviços
como blocos de montagem, nos quais as camadas de serviço podem depender de
outras camadas de serviços. Isto permite o reuso de lógica de aplicações
completamente diferentes para compor uma nova aplicação, o que é possível
somente se outros princípios como o baixo acoplamento e isolamento da regra de
negócio forem aplicados corretamente. Uma grande vantagem da aplicação deste
conceito é que diferentes aplicações podem compartilhar de uma mesma
infraestrutura de serviços.

\subsection{Microserviços}

Microserviços são um formato de desenvolvimento de sistemas que trata o
problema de isolamento dos domínios da aplicação por meio da distribuição
da lógica entre pequenos processos independentes que trabalham em conjunto
para a entrega da funcionalidade final esperada.

Esta arquitetura surgiu de práticas aplicadas por empresas de escala global que
operam na internet, muito antes de haver uma classificação para tal. Estas
práticas se enquadram como um subconjunto dos conceitos de \ac{SOA}, mas se
diferenciam em alguns pontos das tecnologias tradicionais por enforçar práticas
que focam na entrega rápida e na produtividade dos times envolvidos com
o produto.

Segundo \citeonline{Newman2015}, arquiteturas de microserviços não se limitam
somente a arquitetura de software, se estendendo sobre a forma como são
realizadas as entregas de novas versões, no gerenciados os ambientes
de operação, no versionamento do software como um todo e no armazenamento de
dados da aplicação, como apresentado na figura \ref{fig:venn-microservices},
que apresenta a i intersecção de conceitos que trabalham em conjunto para criar
os microserviços.

\begin{figure}[H]
	\centering
	\caption{Diagrama de conceitos de Microserviços}
	\includegraphics[width=1.0\textwidth]{figuras/venn-microservices.png}

	\label{fig:venn-microservices}
	\footnotesize Fonte: \citeonline{Goldsmith2015}
\end{figure}

O baixo acoplamento neste formato de arquitetura é levado um passo a frente,
realizando a separação de domínios da aplicação por meio de isolamento dos
executáveis e do código que compõe os serviços, permitindo que serviços
sejam escritos em linguagens diferentes, caso seja necessário. Isto é útil
quando existe a necessidade de utilizar tecnologias diferentes em um
serviços mais crítico, por motivos de performance, por exemplo.

O versionamento e a entrega de atualizações neste modelo de arquitetura
também se diferencia, por exigir que o versionamento dos serviços sejam
independentes entre si. Isto permite que entregas de novas funcionalidades
sejam realizadas independentemente de outros serviços, aumentado a
produtividade dos times envolvidos, uma vez que cada um pode realizar entregas
sem depender de prazos e atrasos de outras equipes.

Este processo só funciona em escala com ferramentas de apoio bem definidas,
como ferramentas de \ac{CI} e \ac{CD} aplicadas, além de times serem compostos
por integrantes multidisciplinares. Muitos dos conceitos aplicados a
microserviços tiveram algumas raízes nas metodologias ágeis, como Scrum, XP e
Lean, o que torna a aplicação desta arquitetura nos produtos de uma empresa
uma alteração na forma como a própria empresa trabalha.

Segundo \citeonline{Newman2015}, esta nova forma de trabalho otimiza o processo
de desenvolvimento, uma vez que os microserviços normalmente tem sua base de
código separada dos outros serviços, permitindo que o desenvolvedor realize
alterações sem o risco de afetar outras áreas da aplicação.

Os benefícios na produtividade e na qualidade do software entregue pode
ser claramente vista na tese de mestrado de \citeonline{Lopes2015}, onde
ele apresenta um estudo de caso de migração da arquitetura dos produtos
de uma empresa para microserviços, na qual houveram aumentos significativos na
velocidade de entrega de software funcional e na diminiuição dos bugs
encontrados em produção.

\section{REST}

\ac{REST} é um padrão de criação de \ac{APIs} que vem ganhando mercado com a
adoção em massa por grandes empresas da internet. Ele define algumas práticas
para estruturação das \ac{URLs} por meio da utilização do que ele chama de
recursos, que são elementos da aplicação responsáveis por armazenamento ou
tomada de ações, que podem ser isolados a nível da regra de negócio que os
engloba. Segundo \citeonline{Saudate2013}, este padrão pode ser considerado
como uma aplicação de \ac{APIs} orientadas a recursos. Alguns exemplos
de recursos são usuários, \emph{tickets} de um sistema de controle de bugs e
notas fiscais de um software de \ac{ERP}.

Cada recurso dentro da \ac{API} pode ter um conjunto de ações que podem ser
aplicadas sobre ele, que são apresentadas por meio de métodos \ac{HTTP} ou
por sub-\ac{URLs} para ações mais específicas. O conjunto de ações mais
comuns é o \ac{CRUD}, as quais são operações básicas de manipulação de dados,
que são traduzidos para os métodos \emph{POST}, \emph{GET}, \emph{PUT} e
\emph{DELETE} para utilização nas \ac{APIs}.

Um excelente exemplo de aplicação, como apresentado por
\citeonline{Richardson2007}, é a \ac{API} da aplicação S3 da Amazon, que é
um serviço de armazenamento de arquivos e dados. Nesta aplicação, são
utilizados dois recursos principais, os \emph{Buckets} e os \emph{Objects},
onde \emph{Objects} são as unidades de armazenamento de dados e \emph{Buckets}
são containers que contém multiplas unidades de armazenamento, sendo cada
um destes recursos manipulados por \ac{APIs} dedicadas para criação e
manipulação dos mesmos.

Como pode ser visualizado no artigo de  \citeonline{Ignacio2009}, o padrão
\ac{REST} funciona muito bem em conjunto com microserviços, por aplicar algumas
práticas de separação de funcionalidades por meio de isolamento de recursos
relacionados. No estudo de caso deste artigo, é apresentado a criação de uma
\ac{API} no estilo \ac{REST} de pesquisa, na qual o resurso apresentado é o
objeto alvo da pesquisa, o qual é repassado para microserviços que funcionam
como intermediários para busca em motores de busca disponíveis no mercado.

\section{MESSAGE QUEUES}

\subsection{Brokered}

\subsection{Brokerless}

\section{DESCOBERTA DE SERVIÇOS}

\section{CONTAINERS LINUX}

\section{BANCO DE DADOS}

Outra prática aplicada que visa diminuir o acoplamento das rotinas do sistema
é o isolamento da base de dados de cada domínio de aplicação. A recomendação é
cada serviço ter sua própria base de dados e ser o único responsável por ela,
impedindo assim que outros serviços realizem alterações indevidas na base de
dados de outros serviços. A figura \ref{fig:db-monolith-microservices}
apresenta um diagrama comparativo do armazenamento de dados de uma arquitetura
monolítica e de microserviços:

\begin{figure}[H]
	\centering
	\caption{Armazenamento de dados em monolítos e microserviços}
	\includegraphics[width=1.0\textwidth]{figuras/decentralised-data.png}

	\label{fig:db-monolith-microservices}
	\footnotesize Fonte: \citeonline{Fowler2016}
\end{figure}
