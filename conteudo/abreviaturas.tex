% Esse é o arquivo de variáveis do estilo da SOCIESC-IST
% Quando definir e alterar as suas variáveis, observe atentamente a caixa
%  em que o texto está escrito.

\acro{API}[API]{\emph{Application Programming Interface}}
\acrodefplural{API}[APIs]{\emph{Application Programming Interfaces}}
\acro{BD}[BD]{Banco de Dados}
\acro{CD}[CD]{\emph{Continuous Delivery}}
\acro{CI}[CI]{\emph{Continuous Integration}}
\acro{CRUD}[CRUD]{\emph{Create Retrive Update Delete}}
\acro{DW}[DW]{\emph{Data Warehouse}}
\acro{ECP}[ECP]{Bacharelado em Engenharia de computação}
\acro{ERP}[ERP]{\emph{Enterprise Resource Planning}}
\acro{HTTP}[HTTP]{\emph{HyperText Transfer Protocol}}
\acro{JSON}[JSON]{\emph{JavaScript Object Notation}}
\acro{MQ}[MQ]{\emph{Message Queues}}
\acrodefplural{MQ}[MQs]{\emph{Message Queues}}
\acro{REST}[REST]{\emph{REpresentational State Transfer}}
\acro{RPC}[RPC]{\emph{Remote Procedure Call}}
\acro{SGBD}[SGBD]{Sistema de Gerência de Banco de Dados}
\acro{SOA}[SOA]{\emph{Service Oriented Architecture}}
\acro{SOAP}[SOAP]{\emph{Simple Object Access Protocol}}
\acro{SQL}[SQL]{\emph{Structured Query Language}}
\acro{UNISOCIESC}[UNISOCIESC]{Sociedade Educacional de Santa Catarina}
\acro{URL}[URL]{\emph{Uniform Resource Locator}}
\acrodefplural{URL}[URLs]{\emph{Uniform Resource Locators}}
\acro{XML}[XML]{\emph{eXtensible Markup Language}}
\acro{WSDL}[WSDL]{\emph{Web Services Description Language}}
