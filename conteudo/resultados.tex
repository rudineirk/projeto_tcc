\chapter{RESULTADOS OBTIDOS}
\label{chp:resultados}

Neste capítulo são apresentados os resultados deste trabalho, abordando os
ganhos de produtividade com esta arquitetura, o gerenciamento dos
microsserviços, a performance, escalabilidade e tolerância a falhas do sistema.

\section{PRODUTIVIDADE}

Um dos grandes problemas da versão anterior do sistema era a dificuldade da
manutenção e adição de agregar novas funcionalidades no sistema. A aplicação
de padrões de projeto e de arquitetura de \emph{software} consistentes,
como a arquitetura de microsserviços, tem como objetivo diminuir o esforço
e conhecimento requerido para realização de alterações no código
de uma aplicação.

Conforme pode ser visualizado na Figura \ref{fig:prod-dev}, no modelo antigo
do sistema, a criação de uma nova \ac{API} de cadastro simples consumiria
10 horas da equipe de desenvolvimento, considerando as fases de planejamento,
desenvolvimento, testes e liberação da aplicação para produção. Com a
utilização da arquitetura de microsserviços, este tempo foi reduzido para
menos de 2 horas. Esta redução se deve a utilização da arquitetura de
microsserviços, por ser necessário escrever menos código e não depender
de entregas de outras partes do sistema para levar um módulo a produção.

\begin{figure}[H]
	\centering
	\caption{Ganhos de produtividade na utilização de microsserviços}
	\includegraphics[width=0.7\textwidth]{figuras/prod-dev.png}

	\label{fig:prod-dev}
	\footnotesize Fonte: \fonteOAutor
\end{figure}

Para reduzir a carga de manutenção das ferramentas utilizadas no
desenvolvimento, foi escolhido inicialmente a utilização de único
repositório para todo o código do sistema. Este formato de utilização de
repositórios é chamada de \emph{monorepo}. A principal vantagem deste
formato é o aumento da produtividade do desenvolvedor por não ser necessário
acessar diversos repositórios para visualizar todo o código que uma aplicação
utiliza. Outra grande vantagem é a possibilidade realizar alterações que afetam
múltiplos serviços com um único \emph{commit}, ou seja, dentro de um mesmo
conjunto de alterações de código.

Um erro de conceitualização de \emph{monorepos}, é o fato de o código de
funcionalidades diferentes estar dentro de um mesmo repositório categoriza
a aplicação como monolítica, o que não é correto, pois é possível
realizar esta separação em nível de arquitetura de \emph{software}. Isto é
demostrado na prática por este modelo aplicado em empresas que são
referências na área de desenvolvimento de \emph{software}, como a Google,
o Facebook e o Twitter, as quais mantém todo o código da empresa em um
único repositório.

Existem alguns pontos negativos com relação a este formato, pois há
algumas dificuldades para manutenção de versões anteriores de produtos,
que é uma situação comum quando é liberado uma nova versão e a anterior
ainda não encerrou seu ciclo de vida, ou seja, ainda é necessário a
realização de correções de \emph{bugs} na versão anterior. Uma alternativa
a este formato seria utilizar um repositório por serviço ou um repositório
por produto, mas as vantagens de utilizar \emph{monorepos} no lugar de
diversos repositórios ainda prevaleceu para este sistema.

Para facilitar a entrega rápida de \emph{software}, foi aplicado ferramentas de
construção dos binários dos serviços de forma automatizada, que disponibilizam
o \emph{software} para aplicação em produção a cada conjunto de alterações que é
enviado para o servidor de controle de versões do código. Esta técnica é
chamada de \emph{Continuous Delivery}, onde as últimas alterações de código
são disponibilizadas para instalação em um servidor em produção imediatamente
após a construção do binário, mas não necessariamente é instalado
automaticamente a última versão no servidor que roda o \emph{software},
podendo ser aplicadas atualizações com uma frequência menor.

\begin{figure}[H]
	\centering
	\caption{Tempo de abientação de novos desenvolvedores}
	\includegraphics[width=0.7\textwidth]{figuras/prod-ambientacao.png}

	\label{fig:prod-ambientacao}
	\footnotesize Fonte: \fonteOAutor
\end{figure}

Os resultados da aplicação desta arquitetura e das ferramentas de apoio,
utilizadas para simplificar o processo de desenvolvimento, podem ser
medidas com base no tempo necessário para um novo desenvolvedor começar
a produzir novas funcionalidades, em uma velocidade similar aos outros
membros da equipe. Conforme pode ser visualizado na Figura
\ref{fig:prod-ambientacao}, o tempo no sistema anterior era relativamente
alto, um valor próximo a 5 meses, devido aos problemas arquiteturais e de
tecnologias aplicadas.

No novo sistema, esse valor cai para pouco mais de um mês. Esta métrica
leva em conta que o novo membro da equipe não conhece o sistema, mas que
já tenha experiência com a linguagem de programação utilizada. A amostragem
desta métrica foram dois funcionários, que após o treinamento na linguagem
de programação, começaram a desenvolver funcionalidades para o sistema,
utilizando a arquitetura de microsserviços. Na próxima seção é apresentado
as ferramentas de gerenciamento dos serviços em produção.

\section{GERENCIAMENTO DOS SERVIÇOS}

Para gerenciamento dos containers Docker, foi utilizada a ferramenta Swarm,
conforme descreve \citeonline{Matthias2015}, é uma solução de \emph{clusters}
nativa do Docker, que permite criação de serviços com base em containers de
forma simplificada. Essa ferramenta é responsável por baixar, instalar e
manter os serviços rodando, além de permitir o gerenciamento dos mesmos.
Como o Docker e o Swarm tem somente ferramentas de gerenciamento via linha de
comando, foi adicionado uma interface de gerenciamento web chamada Portainer,
que pode ser vista na Figura \ref{fig:portainer-containers}.

\begin{figure}[H]
	\centering
    \caption{Gerenciamento dos containers Docker}
	\includegraphics[width=1.0\textwidth]{figuras/portainer-containers.png}

	\label{fig:portainer-containers}
	\footnotesize Fonte: \fonteOAutor
\end{figure}

O Portainer permite visualizar o consumo de recursos e o status das instâncias
dos serviços, como é demostrado na Figura \ref{fig:portainer-containers}.
A grande vantagem da utilização de um sistema como esse, é o fato de ser
possível alguém que não conheça em detalhes o funcionamento do Docker gerenciar
os serviços do sistema. Além disto, esta interface simplifica a identificação
de problemas e falhas nos serviços que compõe o sistema.

\begin{figure}[H]
	\centering
    \caption{Monitoramento de consumo de recursos do container}
	\includegraphics[width=0.7\textwidth]{figuras/portainer-graph.png}

	\label{fig:portainer-graph}
	\footnotesize Fonte: \fonteOAutor
\end{figure}

O Swarm permite também a criação de \emph{clusters} de máquinas, o que
possibilita a distribuição de instâncias de serviços entre diferentes
computadores dentro do \emph{cluster} de forma transparente. Do ponto de
vista de gerenciamento dos containers, o serviço é executado no
Swarm, não importando em que máquina física ele está instalado. Como pode
ocorrer falhas na máquina física ou nos próprios serviços, foram estudadas
algumas formas de tolerância a estas falhas no sistema, que são apresentadas
na seção seguinte.

\section{TOLERÂNCIA A FALHAS}

Um dos pilares de sistemas distribuídos é que nenhum sistema está totalmente
seguro de falhas, o que pode ser feito é mitigar estas falhas. Um
\emph{cluster} Swarm rodando em mais de uma máquina física tem algumas
funcionalidades que auxiliam para garantir que um serviço esteja o máximo
de tempo possível em execução, independente de falhas em uma instância do
serviço ou na máquina em que ele está executando. Por distribuir as instâncias
de um serviço entre as máquinas do cluster, caso ocorra uma falha em uma das
máquinas ou em uma das instâncias do serviço, ele precisa somente desabilitar
a instância com problemas para eliminar o ponto de falha.

Em nível de serviço, foi adicionado uma forma manutenção das conexões com
outros componentes, que permite que outros componentes sejam reiniciados e
a conexão é reestabelecida em alguns milissegundos. Isto se deve ao RabbitMQ,
pois ele tem algumas garantias quanto a tolerância a falhas, ele faz a
persistência de mensagens não processadas para o disco, caso ocorra alguma
falha brusca, como uma queda de energia, ao iniciar novamente, as mensagens
enfileiradas para processamento são restauradas para a fila. Esta persistência
não foi habilitada somente para chamadas \ac{RPC}, e consequentemente
chamadas de \acp{API}, porque a conexão de rede com o remetente não é mantida,
o que elimina a utilidade das mensagens.

\section{PERFORMANCE}

Na arquitetura de microsserviços, a performance individual de cada parte
do sistema deixa de ser o foco, sendo as análises de performance realizadas
em nível de sistema. Otimizações pontuais não são negligenciadas, mas é
possível contornar problemas de performance em produção aumentando a
quantidade de instâncias de um serviço em execução. Isto permite que
seja reduzida a pressão sobre a equipe de desenvolvimento quanto à correções
de problemas de performance, uma vez que estas correções não precisam ser
aplicadas imediatamente.

Para alterar a quantidade de instâncias executando no Swarm, é possível utilizar
o Portainer para alterar a quantidade de instâncias de um serviço que está
rodando no Swarm, exemplificado na Figura \ref{fig:portainer-replicas}.
Para aumentar a quantidade de instâncias de um serviço utilizando esta
ferramenta, é preciso alterar o campo \emph{Replicas} e salvar as novas
configurações. Dentro de alguns segundos, o Swarm deve iniciar os containers
das novas instâncias do serviço automaticamente.

\begin{figure}[H]
	\centering
	\caption{Gerenciamento das instâncias de um serviço}
	\includegraphics[width=0.7\textwidth]{figuras/portainer-replicas.png}

	\label{fig:portainer-replicas}
	\footnotesize Fonte: \fonteOAutor
\end{figure}

Quando a instância de um serviço inicia, ela automaticamente se registra
no RabbitMQ. A partir deste registro, o RabbitMQ começa a distribuir a carga
de trabalho automaticamente entre as instâncias do serviço. No caso de
serviços que não utilizam a \ac{MQ} como meio de entrada, como o
\emph{API Gateway}, é utilizado o endereço \ac{DNS} fornecido pelo Swarm
para cada serviço, o qual realiza a distribuição das conexões entre as
instâncias do serviço no modelo \emph{round-robin}, que entrega o
endereço de rede de um dos containers rodando a cada requisição, alternando
entre eles.

O sistema como um todo, da forma como foi desenhado, atende os requisitos
apresentados para o projeto, mas a estrutura interna dos microsserviços
apresentou alguns problemas, como alto acoplamento e a comunicação síncrona
entre os componentes. O alto acoplamento entre os componentes internos do
serviço causaram alguma dificuldade de realização de testes unitários, por
ser necessário a criação de componentes simulados para realização dos testes.
Outro problema foi a comunicação síncrona entre os componentes, que
interferiu diretamente na performance do serviço, devido a cada instância do
serviço ficar bloqueada até a finalização do processamento de uma requisição.

Para resolver este problema, uma alternativa estudada seria reestruturação
dos serviços utilizando uma arquitetura de atores. Os atores são componentes
da aplicação que se comunicam entre si por meio de passagem de mensagens,
similar a comunicação via \ac{MQ} entre serviços. Este formato tem a vantagem
que os atores podem ser executados em diferentes \emph{threads} ou processos,
além de permitir a criação de testes unitários somente com a simulação das
mensagens de entrada e validação dos retornos. Isto está planejado para ser
realizado a curto prazo, pois as melhorias na manutenção do código e a
redução de uso de recursos de cada serviço é significativa.

\section{TESTES COM AS APIS}

Para demonstração do funcionamento da arquitetura, foi desenvolvido um
serviço de cadastro de usuários, utilizando a arquitetura de microsserviços.
Este serviço expõe seus métodos via \ac{MQ}, utilizando o RabbitMQ como server.
Os métodos do serviço foram mapeados, via \emph{API Gateway}, para uma
\ac{API} \ac{HTTP}, para uso de terceiros e para acesso via interface.

Esta \ac{API} utiliza o padrão \ac{CRUD}, utilizando como recurso o usuário,
tendo como campos o usuário de acesso, a senha, o nome (separados
em primeiro e último nome) e o e-mail do usuário. Na Figura
\ref{fig:test-create} é possível verificar a criação de um usuário, que
foi realizada utilizando o método \emph{POST} do protocolo \ac{HTTP}.

\begin{figure}[H]
	\centering
	\caption{Criação de um usuário via API}
	\includegraphics[width=0.7\textwidth]{figuras/test-create.png}

	\label{fig:test-create}
	\footnotesize Fonte: \fonteOAutor
\end{figure}

A \ac{API} já possui validação de campos, que verifica o formato dos dados
e a presença de campos obrigatórios. Um exemplo de uma chamada da \ac{API}
com campos faltando pode ser visualizada na Figura \ref{fig:test-validator},
onde é feito uma requisição de criação de um usuário sem os campos de nome
e e-mail do usuário. Essa requisição gera uma mensagem de erro que descreve
quais campos não foram preenchidos, que pode ser utilizado para exibição
na interface, ou para terceiros que estão se integrando com o sistema
conseguirem identificar falhas na integração.

\begin{figure}[H]
	\centering
	\caption{Validação dos dados da requisição para a API}
	\includegraphics[width=0.7\textwidth]{figuras/test-validator.png}

	\label{fig:test-validator}
	\footnotesize Fonte: \fonteOAutor
\end{figure}

Após criado o usuário, as informações deste usuário ficam disponíveis na
\ac{API}, sendo necessário somente adicionar o identificador único do usuário
no final da \ac{URL}. Isto é necessário para a \ac{API} identificar qual
item do recurso, neste caso um usuário da lista de usuários, o cliente da
\ac{API} está tentando acessar. Na Figura \ref{fig:test-get-single} é exibido
o resultado da busca das informações de um usuário, na qual foi utilizado o
método \ac{HTTP} \emph{GET}.

\begin{figure}[H]
	\centering
	\caption{Busca das informações de um usuário via API}
	\includegraphics[width=0.7\textwidth]{figuras/test-get-single.png}

	\label{fig:test-get-single}
	\footnotesize Fonte: \fonteOAutor
\end{figure}

Caso o cliente da \ac{API} precise da lista de usuários do sistema, ela pode
ser acessada utilizando a \ac{URL} base do recurso. Um exemplo de listagem
de usuários na respectiva \ac{API} pode ser visualizado na Figura
\ref{fig:test-get-all}.

\begin{figure}[H]
	\centering
	\caption{Busca da lista de usuários via API}
	\includegraphics[width=0.7\textwidth]{figuras/test-get-all.png}

	\label{fig:test-get-all}
	\footnotesize Fonte: \fonteOAutor
\end{figure}

Neste capítulo foram apresentados os resultados de produtividade, as
ferramentas de gerenciamento do sistema e alguns testes práticos
com uma \ac{API} implementada utilizando a arquitetura de microsserviços.
Os resultados apresentados, com relação a produtividade e velocidade de
entrega de software funcional, demonstraram uma melhoria significativa
com relação ao sistema antigo, que utilizava uma arquitetura monolítica.
A utilização de ferramentas como Docker, Swarm e Portainer facilitaram
a visualização do estado do sistema e a identificação de problemas,
além de simplificar a manutenção dos serviços. No próximo capítulo é
concluido o trabalho, revisando alguns pontos importantes e fazendo
uma análise do resultado do trabalho, finalizando com considerações
de melhorias futuras para a arquitetura.
