 \begin{quadro}[htbp]
\centering
\caption{Os seis passos do D6 \emph{Framework}}
\label{tab:exTabela}
%\vspace{0.4cm}
\begin{tabular}{|l|l|l|} \hline
	Passos  &  Descrição  & Característica  \\
	\hline                                         \hline  
	1D &  Objetivos de negócio & Definir as metas que o sistema precisa realizar   &  \hline 
	2D & Comportamento Alvo  &   Indica o que a organização quer que o usuário faça &  \hline 
	3D & Jogadores &  Conhecer os tipos de jogadores   &  \hline 
	4D &  Atividades em \emph{Loops} & Um jogo gamificado pode ser pensado como um laço & \hline 
	5D & Diversão & Não esquecer da motivação & \hline 
	6D & Ferramentas apropriadas & Utilizar ferramentas convenientes para o projeto     &      	
	\hline
\end{tabular}   

\\ \footnotesize Fonte: o Autor
\end{quadro}

%2d delinea os comportamentos alvo: o segundo passo consiste em determinar o que a organização quer que o usuário faça.
%3d Descrevendo os jogadores: para o terceiro passo, é importante conhecer os jogadores, a fim de criar um sistema eficaz.
%4d Conceber Atividades em \emph{Loops} (laços): os elementos de núcleo de um jogo gamificado podem ser pensados como um laço. Um jogo tem \emph{loops}, ou estruturas que são repetitivas e cursivas, mas que acabam em resultados diferentes.
% 5D - Não se esquecer  da diversão: não se deve perder de vista o divertimento, o interessante. Um dos elementos mais importantes da \emph{gamification}.
% 6D - Implantar as ferramentas apropriadas: usar as ferramentas certas para projetar o jogo também é importante.